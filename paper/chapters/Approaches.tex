\section{Approaches}\label{sec:Approaches}
This chapter explains our aproach in solving the task. First of all it gives a short introduction of how we are going to solve it. Then it explains the two approaches we performed to solve this problem. \\
We want to find the language for every word inside a tweet using mashine learning algorithms.  We try two different approaches. The first is the Linear Regression. Our approach ignores the sentence in its entirety and looks only at distinct words. Using character-level uni- and bigrams as features will the be used to train the Linear Regression.\\
The second approach is the using the word bigram frequency count proposed in the paper Detecting Code-Switches using Word Bigram Frequency Count\cite{bigramFrequencyCount}. This approach looks at the whole sentence not only at a single word. 
\subsection{Linear Regression}
The Linear Regressions goal is it to look at a single word and decide which language it is. It works on a character-level N-Gram basis. The unigrams are every character used in the tweets, and the bigrams is a combination of every character with every character. Next we looked up every distinct word in the provided tweets. Aftwerwards we wrote down a table which showed for each word which uni- and bigrams it contains. Additional we provided the table with the language from which the word is. This way we have a table with one column containing each word. Then one column for each possible bigram and if each word contains this bigram. Lastly a column with the label from which language the word is from. We havent used trigrams since the paper \cite{multipleLanguages} already showed that trigrams wont improve the machinelearning algorithm. The next improvement is reached when using nGrams in the length of the words. But since the trigram table would have had required around 20Gbyte we did not perform this step either. The nGrams in length of words would have been resulting in a hardware problem.\\
This table is then useful in combination with the Python libraries Pandas and Sklearn. We used pandas to read the table and performed the Linear Regression using sklearn.\\
\subsection{Word Bigram Frequency Count}
