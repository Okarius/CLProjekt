\section{Data}\label{sec:Data}
The data provided by \cite{workshop} contains english and spanish tweets. For each tweet are provided with how to split the words and which language this word is. Tweets contain a lot of special characters. Thus the data not only has labels for english \textit{lang1} and spanish \textit{lang2}. But also \textit{other} for emoticons, nicknames or gibberish. With example ":)", "@Ody12", or "Zaaaas" are labeled as other. The lable \textit{mixed} describes if a word contains both, spanish and english words, "ClutterDesordenLook" is such a word.  The next label is \textit{ne}, ne is a name entity. Those are proper names refering to people, places, organizations, locations or titles. The difficulty for those is that they could span above multiple words "West Coast" for example. \\
The data provides information of how to split the tweet into words. This is important in cases where punctuation marks or emoticons are directly at a word without whitespace in between with example "HEY!:)". \\
Lets say the data gives us the tweet: "@snapchateame No! Jason you look bueno!:). Next thing provides the given data the information to split the tweet into words to look like this: "snapchateame, No, Jason, you, look, bueno, :)". And finaly the data tells us which label every word gets "mixed, ambiguous, ne, lang1, lang1, lang1, lang2, other". \\
